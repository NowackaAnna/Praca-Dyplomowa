\documentclass[a4paper,12pt,reqno]{article}
%----------------------------------------------------------
% This is a sample document for the AMS LaTeX Article Class
% Class options
%        -- Point size:  8pt, 9pt, 10pt (default), 11pt, 12pt
%        -- Paper size:  letterpaper(default), a4paper
%        -- Orientation: portrait(default), landscape
%        -- Print size:  oneside, twoside(default)
%        -- Quality:     final(default), draft
%        -- Title page:  notitlepage, titlepage(default)
%        -- Start chapter on left:
%                        openright(default), openany
%        -- Columns:     onecolumn(default), twocolumn
%        -- Omit extra math features:
%                        nomath
%        -- AMSfonts:    noamsfonts
%        -- PSAMSFonts  (fewer AMSfonts sizes):
%                        psamsfonts
%        -- Equation numbering:
%                        leqno(default), reqno (equation numbers are on the right side)
%        -- Equation centering:
%                        centertags(default), tbtags
%        -- Displayed equations (centered is the default):
%                        fleqn (equations start at the same distance from the right side)
%        -- Electronic journal:
%                        e-only
%------------------------------------------------------------
% For instance the command
%          \documentclass[a4paper,12pt,reqno]{amsart}
% ensures that the paper size is a4, fonts are typeset at the size 12p
% and the equation numbers are on the right side
%
\usepackage{amsfonts}
\usepackage{graphicx}
\usepackage{geometry}
\usepackage{color}
\usepackage{amssymb,amsmath}
\usepackage{polski}
\usepackage[T1]{fontenc}
\usepackage[utf8]{inputenc}
\usepackage{caption}
\geometry{margin=1.1in}
\usepackage{wrapfig}
\usepackage{lipsum}  
\usepackage{listings}
\usepackage[toc,page]{appendix}

\definecolor{codegreen}{rgb}{0.5, 0.09, 0.09}
\definecolor{codegray}{rgb}{0.5,0.5,0.5}
\definecolor{codepurple}{rgb}{0.58,0,0.82}
\definecolor{backcolour}{rgb}{0.94,0.94,0.94}
\definecolor{gray}{rgb}{0,0.6,0}

\lstdefinestyle{mystyle}{
    backgroundcolor=\color{backcolour},  
    commentstyle=\color{codegreen},
    keywordstyle=\color{blue},
    numberstyle=\tiny\color{codegray},
    stringstyle=\color{codepurple},
		basicstyle=\footnotesize\fontfamily{cmtt}\selectfont,
    breakatwhitespace=false,         
    breaklines=true,
    captionpos=b,
		language=C++,
    keepspaces=true,                 
    numbers=left,                    
    numbersep=5pt,                  
    showspaces=false,                
    showstringspaces=false,
    showtabs=false,                  
    tabsize=2
}
 
\lstset{style=mystyle}
\lstset{literate=%
    *{0}{{{\color{gray}0}}}1
    {1}{{{\color{gray}1}}}1
    {2}{{{\color{gray}2}}}1
    {3}{{{\color{gray}3}}}1
    {4}{{{\color{gray}4}}}1
    {5}{{{\color{gray}5}}}1
    {6}{{{\color{gray}6}}}1
    {7}{{{\color{gray}7}}}1
    {8}{{{\color{gray}8}}}1
    {9}{{{\color{gray}9}}}1
}
%------------------------------------------------------------
\begin{document}

%\begin{figure}[h]
%	\centering
%		\includegraphics[width=0.40\textwidth]{logo.pdf}
%\end{figure}


\begin{center}

\thispagestyle{empty}

%UNIWERSYTET WROCŁAWSKI\\
\Large 
Uniwersytet Wrocławski\\
Wydział Fizyki i Astronomii\\
\vspace{0.8cm}
\vspace{1.8cm}

\Large Anna Boratyńska \\
\vspace{3.2cm}
\Large Aplikacja mobilna służąca do rejestracji treningu biegowego oraz posiadająca funkcję dziennika treningowego dla biegaczy. \\
\vspace{1.5cm}
Mobile application to recording runner's trening with training's diary function.
\end{center}
\vspace{3.7cm}
\begin{flushright}
\large{ Praca inżynierska na kierunku \\Informatyka Stosowana i Systemy Pomiarowe \\}
\vspace{0.5cm}
\large{ Opiekun \\ dr hab. Maciej Matyka, prof. UWr}
\end{flushright}
\vspace{2.2cm}

\begin{center}
\large Wrocław, 2021
\end{center}
\newpage

\tableofcontents

\newpage

\begin{flushleft}
	\Large \textbf{Streszczenie}
\end{flushleft}
\vspace{1cm}
Celem niniejszej pracy jest stworzenie aplikacji dla biegaczy. Ma ona służyć zarówno do rejestracji treningu biegowego, jak i jako dziennik treningowy, do którego można zapisywać zarejestrowane treningi, dodawać jednostki uzupełniające, przeglądać je i usuwać. Najważniejszą cechą tej aplikacji ma być prostota użytkowania. Jest ona zaprojektowana m.in. z myślą o osobach, które dopiero zaczynają swoją przygodę z bieganiem i nie chcą jeszcze inwestować w drogi sprzęt - zegarki z modułem GPS. Ma ona też służyć osobom, które ze względów prywatnych powinny mieć zawsze przy sobie telefon - możemy mieć zarówno sprzęt do biegania, jak i możliwość komunikacji w jednym urządzeniu. Jest dobrym, czasowym zastępstwem, gdy mamy zgubiony/uszkodzony zegarek (do biegania). Warto zaznaczyć, że aplikacja specjalnie jest stworzona jako proste w obsłudze rozwiązanie tymczasowe, ponieważ częste, długookresowe bieganie z telefonem nie jest zalecane.\\ \\
Aplikacja działa od razu po pobraniu, bez zakładania konta/logowania. Posiada lokalną bazę danych w której przechowywane są zarejestrowane treningi biegowe i dodane uzupełniające. Oprócz tego korzysta ona z modułu GPS oraz stopera. Wykorzystane narzędzia: Android Studio, SQLite/Firebase.
\end{document}